
% Thesis Abstract -----------------------------------------------------


%\begin{abstractslong}    %uncommenting this line, gives a different abstract heading
\begin{abstracts}        %this creates the heading for the abstract page

%{\color{red} Write me last ...}

  We study structured covariance matrices in a Gaussian setting for a variety of data analysis scenarios.
  Despite its simplistic nature, we argue for the broad applicability of the Gaussian family through its second order statistics.
  We focus on three types of common structures in the machine learning literature: covariance \emph{functions}, \emph{low-rank} and \emph{sparse inverses} covariances.
  Our contributions boil down to combining these structures and designing algorithms for maximum-likelihood or MAP fitting: for instance, we use covariance functions in Gaussian processes to encode the temporal structure in a gene-expression time-series, with any residual structure generating iid noise.
  More generally, for a low-rank residual structure (correlated residuals) we introduce the \emph{residual component analysis } framework: based on a generalised eigenvalue problem, it decomposes the residual low-rank term given a partial explanation of the covariance.
  In this example the explained covariance would be an RBF kernel, but it can be any positive-definite matrix.
  Another example is the \emph{low-rank plus sparse-inverse} composition for structure learning of GMRFs in the presence of confounding latent variables.
  We also study RCA as a novel link between classical low-rank methods and modern probabilistic counterparts: the geometry of oblique projections shows how PCA, CCA and \emph{linear discriminant analysis} reduce to RCA.
  Also inter-battery factor analysis, a precursor of multi-view learning, is reduced to an iterative application of RCA.
  Finally, we touch on structured precisions of matrix-normal models based on the Cartesian factorisation of graphs, with appealing properties for regression problems and interpretability.
  In all cases, experimental results and simulations demonstrate the performance of the different methods proposed.

\end{abstracts}
%\end{abstractlongs}


% ----------------------------------------------------------------------


%%% Local Variables: 
%%% mode: latex
%%% TeX-master: "../thesis"
%%% End: 
